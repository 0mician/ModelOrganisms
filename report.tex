\documentclass[11pt, a4paper,titlepage]{article}
\usepackage[utf8]{inputenc}
\usepackage[T1]{fontenc}
\usepackage{fixltx2e}
\usepackage{graphicx}
\usepackage{longtable}
\usepackage{float}
\usepackage{wrapfig}
\usepackage{textcomp}
\usepackage{hyperref}
\usepackage[bottom]{footmisc}
\tolerance=1000
\usepackage[left=2.35cm, right=3.35cm, top=3.35cm, bottom=3.35cm]{geometry}
\usepackage[utf8]{inputenc}
\usepackage[greek,english]{babel}
\usepackage{titlesec}
\usepackage{tocbibind}
\makeatletter
\def\@seccntformat#1{%
  \expandafter\ifx\csname c@#1\endcsname\c@section\else
  \csname the#1\endcsname\quad
  \fi}
\makeatother
\begin{document}

\begin{titlepage}
  \begin{center}
    
    \includegraphics[scale=1.5]{Figures/kuleuven_logo.pdf}~\\[4.5cm]
    
    \textsc{\Large Model Organisms}\\[0.5cm]
    
    % Title
    \rule{\linewidth}{0.3mm}\\[0.4cm]
    {\huge \bfseries Test Exam} \\[0.4cm]
    {\large Spring 2015} \\[0.4cm]
    \rule{\linewidth}{0.3mm}\\[1.5cm]
    
    % Author
    \begin{minipage}{0.4\textwidth}
      \begin{center} \large
        \emph{Author:}\\
        Cedric \textsc{Lood}\\
      \end{center}
    \end{minipage}
    
    \vfill
    
    \includegraphics[scale=0.15]{Figures/KUL.jpg}~\\[0.5cm]

    % Bottom of the page
    {\large \today}
    
  \end{center}
\end{titlepage}

\setcounter{tocdepth}{3}

\tableofcontents
\newpage

\section*{Part 1: theory}
\addcontentsline{toc}{section}{Part 1: theory}

To unravel the regulatory system(s) involved in ER stress in
\emph{C. Elegans}, you perform a forward mutagenesis screen to search
for putative regulators of \emph{hsp-4}, a gene encoding a chaperone
protein. For this purpose you made transgenic \emph{[hsp-4::gfp]}
worms that are able to stably express the integrated GFP reporter
under control of the \emph{hsp-4} promoter.
\bigskip

\noindent\texttt{Q:} Which proof-of-principle experiment would be needed to
verify that the reporter strain inded reflects the process you are
interested in?
\smallskip

\noindent\texttt{A:} 
\bigskip

\noindent\texttt{Q:} How would you proceed with the screen to
eventually identify new regulators?
\smallskip

\noindent\texttt{A:} 

\section*{Part 2: practical}
\addcontentsline{toc}{section}{Part 2: practical} 

The Snf1/AMPK/SnRK kinases are a well-conserved family of
AMP-activated kinases in the eukaryotic kingdom where they function as
energy and metabolic sensors. In the yeast \emph{Saccharomyces
  cerevisae} this kinase was studied extensively and found to consist
of a heterotrimeric protein complex with a catalytic $\alpha$-subunit
and a number of regulatory $\beta$-subunits and a $\gamma$-subunit.

\subsection*{Genes}
\addcontentsline{toc}{subsection}{Genes}

\texttt{Q:} Which genes encode these subunits?
\smallskip

\noindent\texttt{A:} 

\subsection*{Phenotypes}
\addcontentsline{toc}{subsection}{Phenotypes}
\texttt{Q:} What are the main phenotypes of yeast cells lacking of
this kinase?
\smallskip

\noindent\texttt{A:} 

\subsection*{Transcription factors}
\addcontentsline{toc}{subsection}{Transcription factors}

\texttt{Q:} Identify at least three transcription factors that are
under control of this kinase complex in yeast.
\smallskip

\noindent\texttt{A:} 

\subsection*{Phosphorylation}
\addcontentsline{toc}{subsection}{Phosphorylation}

\texttt{Q:} To obtain its full activity in yeast, the catalytic
$\alpha$-subunit of AMPK must be phosphorylated. Find the three upstream
protein kinases that conduct this phosphorylation in \emph{S. Cerevisae}.

\subsection*{Deletion mutant(s)}
\addcontentsline{toc}{subsection}{Deletion mutant(s)}

\texttt{Q:} What is the phenotype of the deletion mutant(s) of the
homologue(s) of the catalytic $\alpha$-subunit in \emph{C. Elegans}?
\smallskip

\noindent\texttt{A:} 

\subsection*{Mutant drosophila}
\addcontentsline{toc}{subsection}{Mutant drosophila}

\texttt{Q:} In the fruit fly null mutants for AMPK were obtained by
P-element activity. Give the phenotype of one of these mutants.
\smallskip

\noindent\texttt{A:} 

\subsection*{Zebrafish}
\addcontentsline{toc}{subsection}{Zebrafish}

\texttt{Q:} What is the name of the gene for the AMPK-$\alpha$1
catalytic subunit in zebrafish? Where is this gene expressed in the
14-19 somite embryo and how was this expression visualized? Give a
morpholino sequence for this gene.
\smallskip

\noindent\texttt{A:} 

\subsection*{Other model systems}
\addcontentsline{toc}{subsection}{Other model systems}

\texttt{Q:} You want to find new genes that affect complex formation
of AMPK. Which model system would you use to address this challenge,
and how would you set up the experiments leading to useful results?
\smallskip

\noindent\texttt{A:} 

\bibliographystyle{plain} \bibliography{bib-db}
\end{document}
