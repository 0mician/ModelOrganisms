\documentclass[11pt, a4paper,titlepage]{article}
\usepackage[utf8]{inputenc}
\usepackage[T1]{fontenc}
\usepackage{fixltx2e}
\usepackage{graphicx}
\usepackage{longtable}
\usepackage{float}
\usepackage{wrapfig}
\usepackage{textcomp}
\usepackage{hyperref}
\usepackage[bottom]{footmisc}
\tolerance=1000
\usepackage[left=2.35cm, right=3.35cm, top=3.35cm, bottom=3.35cm]{geometry}
\usepackage[utf8]{inputenc}
\usepackage[greek,english]{babel}
\usepackage{titlesec}
\usepackage{tocbibind}
\makeatletter
\def\@seccntformat#1{%
  \expandafter\ifx\csname c@#1\endcsname\c@section\else
  \csname the#1\endcsname\quad
  \fi}
\makeatother
\begin{document}

\begin{titlepage}
  \begin{center}
    
    \includegraphics[scale=1.5]{Figures/kuleuven_logo.pdf}~\\[4.5cm]
    
    \textsc{\Large Model Organisms}\\[0.5cm]
    
    % Title
    \rule{\linewidth}{0.3mm}\\[0.4cm]
    {\huge \bfseries Test Exam} \\[0.4cm]
    {\large Spring 2015} \\[0.4cm]
    \rule{\linewidth}{0.3mm}\\[1.5cm]
    
    % Author
    \begin{minipage}{0.4\textwidth}
      \begin{center} \large
        \emph{Author:}\\
        Cedric \textsc{Lood}\\
      \end{center}
    \end{minipage}
    
    \vfill
    
    \includegraphics[scale=0.15]{Figures/KUL.jpg}~\\[0.5cm]

    % Bottom of the page
    {\large \today}
    
  \end{center}
\end{titlepage}

\setcounter{tocdepth}{3}

\tableofcontents
\newpage

\section*{Part 1: Theory}
\addcontentsline{toc}{section}{Part 1: Theory}

To unravel the regulatory system(s) involved in ER stress in
\emph{C. Elegans}, you perform a forward mutagenesis screen to search
for putative regulators of \emph{hsp-4}, a gene encoding a chaperone
protein. For this purpose you made transgenic \emph{[hsp-4::gfp]}
worms that are able to stably express the integrated GFP reporter
under control of the \emph{hsp-4} promoter.
\bigskip

\noindent\texttt{Q:} Which proof-of-principle experiment would be needed to
verify that the reporter strain inded reflects the process you are
interested in?
\smallskip

\noindent\texttt{A:} In the lab, the worms will not be under stress
(except if purposefully put in stress conditions). Hence, the
transgenic worms would not display GFP (or only some basal
fluorescence). The first thing need you need to do is then to induce
ER stress to see if GFP is displayed. You also need a negative
control, so as to rule out the expression of GFP under other types of
stress (it needs to be specific to ER stress). Once you have verified
this, you have enough evidence that the GFP is only produced under ER
stress conditions.
\bigskip

\noindent\texttt{Q:} How would you proceed with the screen to
eventually identify new regulators?
\smallskip

\noindent\texttt{A:} You should start by mutagenizing your transgenic
worms. For that you can use EMS. You end up with many mutant strains
which you then need to screen for the specific phenotype you search
for (ER stress response defect, for example less or more
fluorescence). You then have a certain number of candidate mutated
genes that may have an effect on stress response.

You can then proceed to cross the mutants with an Hawaiian strain of
clean background, which have SNPs every 1000 bp. You search for the
SNPs in the offsprings. You should expect to have regions of DNA where
the SNPs count changes dramatically (drops to 0). You can then
introduce a WT cross to see if there is a rescue of the functionality
for those genes.

\section*{Part 2: Practical}
\addcontentsline{toc}{section}{Part 2: Practical} 

The Snf1/AMPK/SnRK kinases are a well-conserved family of
AMP-activated kinases in the eukaryotic kingdom where they function as
energy and metabolic sensors. In the yeast \emph{Saccharomyces
  cerevisae} this kinase was studied extensively and found to consist
of a heterotrimeric protein complex with a catalytic $\alpha$-subunit
and a number of regulatory $\beta$-subunits and a $\gamma$-subunit.

\noindent\underline{Note}: there are 3 names because of naming
conventions in different model organism. Snf1 is unicellular (yeast),
SnRK is in plants, AMPK for about any other organism. These are clues
for the search in the corresponding databases.

\subsection*{Genes}
\addcontentsline{toc}{subsection}{Genes}

\texttt{Q:} Which genes encode these subunits?
\smallskip

\noindent\texttt{A:} The information can be found on the SGD website
in the summary section (the yeast DB has a very strong summary
section, always start from there).  

The active Snf1p kinase complex is a heterotrimeric complex composed
of Snf1p, the catalytic (alpha) subunit; Snf4p, a regulatory (gamma)
subunit; and one of three possible beta subunits (Gal83p, Sip1p, or
Sip2p) which appear to tether Snf1p and Snf4p together and also may
determine substrate specificity of the Snf1p kinase complex. The Ps in
the names refer to proteins, so you can just drop the Ps from the names
(you can also click on the names and it will open the respective gene
pages).

\subsection*{Phenotypes}
\addcontentsline{toc}{subsection}{Phenotypes}
\texttt{Q:} What are the main phenotypes of yeast cells lacking of
this kinase?
\smallskip

\noindent\texttt{A:} SGD, phenotype section. The interesting section
is the classical genetics section, as the results from the other one
(Large scale survey) may not have been verified. You should then look
at the NULL mutations, and as you can see there is a long list of
phenotypes there. They show a common trend in decrease of metabolic
function, which makes sense given the role of these sensor
proteins. Stress resistance, and lifespan are also lowered.

\subsection*{Transcription factors}
\addcontentsline{toc}{subsection}{Transcription factors}

\texttt{Q:} Identify at least three transcription factors that are
under control of this kinase complex in yeast.
\smallskip

\noindent\texttt{A:} The Snf1p kinase complex, which phosphorylates
serine and threonine residues, is essential for regulating the
transcriptional changes associated with glucose derepression through
its activation of the transcriptional activators Cat8p and Sip4p, and
its deactivation of the transcriptional repressor Mig1p. You can
verify that these are indeed Transcription Factors by going to the
corresponding gene pages.

\subsection*{Phosphorylation}
\addcontentsline{toc}{subsection}{Phosphorylation}

\texttt{Q:} To obtain its full activity in yeast, the catalytic
$\alpha$-subunit of AMPK must be phosphorylated. Find the three upstream
protein kinases that conduct this phosphorylation in \emph{S. Cerevisae}.

\noindent\texttt{A:} Snf1p is activated by phosphorylation on
threonine 210 by either Sak1p, Tos3p, or Elm1p. You can verify their
upstream positions through the individual gene pages.

\subsection*{Deletion mutant(s)}
\addcontentsline{toc}{subsection}{Deletion mutant(s)}

\texttt{Q:} What is the phenotype of the deletion mutant(s) of the
homologue(s) of the catalytic $\alpha$-subunit in \emph{C. Elegans}?
\smallskip

\noindent\texttt{A:} You can go through homologene (NCBI) or to the
wormbase site to find the information, for the latter, search the SNF1
and then go to the homology tab. You can find 2 genes: aak-1 and
aak-2. For aak-1, on the phenotype tab (wormbase), you can find one
deletion mutant (remember RNAi is not a deletion mutant, also make
sure it is a deletion and not a SNP). But, there are no strains
available with only aak-1 deleted, so no phenotype can be
observed. For aak-2, there are more details. You can search the
genetics tab of the gene to find the deletion mutants. There are 3
deletion mutants, of which 2 have associated phenotypes (gt33,
ok524). For gt33, you can see the following phenotypes:
\begin{itemize}
\itemsep0em
\item Frequency body bend reduced
\item Organism oxidative stress response hypersensitive
\item Paraquat hypersensitive
\item Protein phosphorylation reduced
\item Suppression of head oscillations defective
\end{itemize} 

\subsection*{Mutant drosophila}
\addcontentsline{toc}{subsection}{Mutant drosophila}

\texttt{Q:} In the fruit fly null mutants for AMPK were obtained by
P-element activity. Give the phenotype of one of these mutants.
\smallskip

\noindent\texttt{A:} The information can be found in the flybase
database, querying for AMP kinase in the quick search. The gene is
AMPK$\alpha$. On the classical alleles (mutagenesis/P-elements), or
the deletions and duplications tabs, you can see that
AMPK$\alpha\delta$39 is interesting and is available from stock. It
has the following phenotypes:
\begin{itemize}
\itemsep0em
\item developmental rate defective (second and third instar larval stage)
\item lethal (P-stage and third instar larval stage)
\item small body (second and third instar larval stage)
\end{itemize} 

\noindent Also, a detailed description is available in the
corresponding tab: Hemizygous larvae are smaller than wild-type larvae
from the late second instar onwards, with the defect becoming more
striking during the third instar. The moult from second to third
instar is delayed by approximately 12 hours in mutant animals, and the
third instar is extended by two days compared to controls. Mutant
animals fail to undergo metamorphosis, dying at the end of the third
larval instar or shortly after, forming abnormal elongated
pupae. Whole body triglyceride levels are significantly decreased in
mutant third instar larvae compared to wild type. Fat body cells are
smaller than in controls in mutant third instar and wandering
larvae. However, homozygous clones of cells in the larval fat body are
equal in size to the surrounding heterozygous cells and starvation has
no differential effect on cell size within the mutant
clones. Homozygous third instar larvae have a pronounced brush border
in the midgut epithelia (as occurs in wild type), but the mutant
midgut epithelial cells show marked vacuolation. The midgut
musculature has a ragged appearance in late mutant third instar
larvae, with both the circular and longitudinal muscles being smaller
than in controls. The thickness of the hindgut muscle fibres is also
reduced in the mutant larvae compared to wild type. Mutant third
instar larvae transferred to dyed food show a similar amount of dyed
food in their gut as control larvae after 24 hours feeding. However,
the mutant larvae show a defect in gut clearance, with significant
amounts of dyed food remaining in their guts even 24 hours after
removal of the dyed food (most wild-type larvae clear the dyed food
from their guts within 10 hours). Peristaltic contractions are not
seen in freshly dissected mutant third larval instar midguts, in
contrast to the spontaneous muscle contraction seen in the majority of
wild-type guts.

\subsection*{Zebrafish}
\addcontentsline{toc}{subsection}{Zebrafish}

\texttt{Q:} What is the name of the gene for the AMPK-$\alpha$1
catalytic subunit in zebrafish? Where is this gene expressed in the
14-19 somite embryo and how was this expression visualized? Give a
morpholino sequence for this gene. 
\smallskip

\noindent\texttt{A:} the information can be found in the zfin
database, querying for AMP kinase in the quick search. The gene is
prkaa1. Go to the expression data (ALL), and there you can see the
14-19 somite, and by clicking on the figure, you can see that they
used mRNA in situ hybridization to vizualize the expression.

The morpholino can be found in the Mutants and Targeted Knockdowns
section of the gene page. MO-prkaa1 is available with the following
sequence:

\begin{verbatim}
5' - AATGAAGAGTTGACTTACCAAAATC - 3'
\end{verbatim}

\subsection*{Other model systems}
\addcontentsline{toc}{subsection}{Other model systems}

\texttt{Q:} You want to find new genes that affect complex formation
of AMPK. Which model system would you use to address this challenge,
and how would you set up the experiments leading to useful results?
\smallskip

\noindent\texttt{A:} No prefered answer here because it depends on the
specific problem you want to study. But, you need to be able to argue
for the model system you select. You could use yeast, create a
reporter strain (like we did for the c. elegans), and then do a
mutagenesis. For this type of question you would typically go for a
lower organism because it is a very basic and well-conserved construct
(housekeeping). If you go to higher organism, you may do so to study
tissues, like brain tissues, eyes, ..

\bibliographystyle{plain} \bibliography{bib-db}
\end{document}
